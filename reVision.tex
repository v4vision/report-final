\documentclass[12pt, a4paper]{article} \pagenumbering{gobble}
\usepackage{gantt} \usepackage{pdflscape} \usepackage{parskip}
\usepackage{afterpage} \usepackage{pbox}
\usepackage[margin=1in]{geometry} \usepackage{multirow,tabularx}
\usepackage{wrapfig}
\usepackage{graphicx}
\usepackage{listings}
\usepackage[dvipsnames]{xcolor}
\usepackage{smartdiagram}
\usesmartdiagramlibrary{additions}
\graphicspath{ {images/} }
\pagenumbering{arabic}
\newcolumntype{Y}{>{\centering\arraybackslash}X}
\renewcommand{\arraystretch}{2}
\definecolor{new-blue}{RGB}{51,153,255}



\begin{document}
  \begin{titlepage}
    \centering
    {\scshape\Large Izmir Institute of Technology\par}
    {\scshape\large Department of Computer Engineering\par}
    {CENG415 Senior Design Project & Seminar 1\par}
    \vspace{2.5cm}
	  {\huge re\sqrt{ision}\par}
	  \vspace{1cm}
	  {\Large Renderscript for Computer Vision\par}
	\vspace{2cm}
	{\Large An{\i}l Can Ayd{\i}n, Onur Temizkan, Ula\c{s} Akdeniz\par}
  {\itshape180201060, 180201004, 180201063\par}
	\vfill
	supervised by\par
	\large{Asst. Prof. Dr.~Mustafa \textsc{\"{O}zuysal}}

	\vfill

% Bottom of the page
	{\large \today\par}
  \end{titlepage}
  \newpage
  \tableofcontents

\newpage
\begin{section}{Description of the Work}
\paragraph{}{%
  Computer vision applications need parallel computation to perform in reasonable response time.
  In order to satisfy that performance, developers use libraries like OpenCL and CUDA which move the computation
  load to GPU. However high performance GPU intensive computation on Android platform is a tough issue because
  of hardware dependencies and incompatibilities of libraries. For instance OpenCL GPU computation library is not
  supported on all Android devices. So, creating hardware independent applications using OpenCL is simply impossible.
  Because of those restrictions mentioned above, Renderscript computation module is presented by Google in 2011\cite{renderscript_release}.
  It is a hardware-independent computation engine that operates at the native level\cite{renderscript_api}. But there is no vision library
  written in Renderscript.
}
\paragraph{}{%
  reVision is an experimental Computer Vision project that strives to prove performance benefits of using Renderscript on Android
  applications. This project includes the performance comparison between Renderscript, OpenCV, and Java implementations on a Computer
  Vision algorithm. For this purpose, Harris Corner and Edge Detector Algorithm\cite{harris_article} has chosen.
}

\paragraph{}{
  Although, Renderscript is presented in 2011, it is not a commonly used option to make computationally intensive tasks.
  It is a considerable approach to use Renderscript for Computer Vision, but there are not enough data to support this idea.
  To provide a reliable analysis, performance of Renderscript on Computer Vision is compared with OpenCV\cite{opencv} which is a de facto standard
  in its domain. Also it is compared with Java, because of both running on CPU and being the main language of Android platform.
}
\end{section}

\newpage

\begin{section}{The Work Plan}
  \begin{subsection}{Overall Strategy}
    \paragraph{}{reVision's iterative development process consists of project management and planning, requirement analysis,
    implementation and testing. Each iteration results in a more optimized version of each implementation.}\\ \\
    Summary of an iteration is described below.
    \paragraph{Planning:}{The scope of the implementations to be written, desired specs and details are discussed.}
    \paragraph{Requirement Analysis:}{The requirements of the implementations, needed arguments and optional specs are discussed.}
    \paragraph{Implementation:}{The sample application for each variation is implemented.
      \begin{itemize}
        \item Obtaining consistent results for each implementation is main objective of every iteration.
        \item Proceed every iteration with phases such as: Design, Implement, Test and Evaluate.
      \end{itemize}
    }
    \paragraph{Testing:}{Sample applications are tested on different devices in different resolutions.}
  \end{subsection}

  \newpage

  \begin{subsection}{Gantt Chart}

  \begin{centering}
  \scalebox{0.7}{
  \begin{gantt}[xunitlength=0.5cm,fontsize=\small,titlefontsize=\small,drawledgerline=true]{21}{32}
    \begin{ganttitle}
      \titleelement{October}{4}
      \titleelement{November}{4}
      \titleelement{December}{4}
      \titleelement{January}{4}
      \titleelement{February}{4}
      \titleelement{March}{4}
      \titleelement{April}{4}
      \titleelement{May}{4}

    \end{ganttitle}
    \begin{ganttitle}
      \numtitle{3}{1}{4}{2}
      \numtitle{1}{1}{4}{1}
      \numtitle{1}{1}{4}{1}
      \numtitle{1}{1}{4}{1}
      \numtitle{1}{1}{4}{1}
      \numtitle{1}{1}{4}{1}
      \numtitle{1}{1}{4}{1}
      \numtitle{1}{1}{4}{1}
    \end{ganttitle}
    \ganttbar[color=purple]{\textbf{Software Development}}{0}{31}
    \ganttbar[color=red]{\textbf{Research / Requirement Analysis}}{0}{28}
    \ganttbar[color=purple]{12.10.15 - 29.04.16}{0}{0}
    \ganttbar[color=red]{\textbf{Design}}{10}{20}
    \ganttbar[color=purple]{14.12.15 - 13.05.16}{0}{0}
    \ganttbar[pattern=crosshatch,color=green]{\textbf{Data Model Design}}{10}{20}
    \ganttbar[color=purple]{14.12.15 - 13.05.16}{0}{0}
    \ganttbar[pattern=crosshatch,color=green]{\textbf{Domain Model Design}}{11}{19}
    \ganttbar[color=purple]{21.12.15 - 13.05.16}{0}{0}
    \ganttbar[color=red]{\textbf{Implementation}}{12}{19}
    \ganttbar[color=purple]{04.01.16 - 20.05.16}{0}{0}
    \ganttbar[pattern=crosshatch,color=green]{\textbf{Renderscript Implementation}}{12}{10}
    \ganttbar[color=purple]{04.01.16 - 18.03.16}{0}{0}
    \ganttbar[pattern=crosshatch,color=green]{\textbf{Java Implementation}}{22}{3}
    \ganttbar[color=purple]{21.03.16 - 08.04.16}{0}{0}
    \ganttbar[pattern=crosshatch,color=green]{\textbf{OpenCV Implementation}}{25}{5}
    \ganttbar[color=purple]{11.04.16 - 12.05.16}{0}{0}
    \ganttbar[pattern=crosshatch,color=green]{\textbf{Test}}{14}{17}
    \ganttbar[color=purple]{18.01.16 - 20.05.16}{0}{0}
  \end{gantt}
  }
  \end{centering}

\end{subsection}

\begin{landscape}% Landscape page
\subsection{Detailed Work Description}
  \subsubsection{Work Package List}
  \scalebox{0.7}{
   \begin{tabularx}{85em}{|*{8}{Y|}}
     \hline
       \textbf{Work Package No} & \textbf{Work Package Title} & \textbf{Type of Activity}\footnotemark[1] & \textbf{Lead Participant No} & \textbf{Lead Participant Short Name} & \textbf{Person weeks}\footnotemark[2] & \textbf{Start Month} & \textbf{End Month} \\
       \hline
       WP-1 & Research and Requirement Analysis & MGT & 1-2-3 & AA-OT-UA & 26 & 1 & 8 \\
       \hline
       WP-2.1 & Data Model Design & SUPP & 1-2-3 & AA-OT-UA & 20 & 3 & 8 \\
       \hline
       WP-2.2 & Domain Model Design & SUPP & 1-2-3 & AA-OT-UA & 19 & 3 & 8 \\
       \hline
       WP-3.1 & Renderscript Implementation & SUPP & 1-2-3 & AA-OT-UA & 10 & 4 & 6 \\
       \hline
       WP-3.2 & Java Implementation & SUPP & 1-2-3 & AA-OT-UA & 3 & 6 & 7 \\
       \hline
       WP-3.3 & OpenCV Implementation & SUPP & 1-2-3 & AA-OT-UA & 5 & 7 & 8 \\
       \hline
       WP-4 & Testing & MGT & 1-2-3 & AA-OT-UA & 18 & 4 & 8 \\
       \hline
        \multicolumn{5}{|c|}{\textbf{TOTAL}} & \textbf{101} & & \\
        \hline
   \end{tabularx}
 }
\footnotetext[1]{\textbf{SUPP} stands for Support activities; \textbf{MGT} stands for Management of the consortium.}
\footnotetext[2]{The total number of person-weeks allocated to each work package.}
\end{landscape}
\subsubsection{Deliverable List}
    \begin{tabularx}{\textwidth}{|*{6}{Y|}}
        \hline
      \textbf{Deliverable No} & \textbf{Deliverable Name} & \textbf{WP No} & \textbf{Nature}\footnote{\textbf{R} = Report, \textbf{P} = Prototype, \textbf{D} = Demonstrator} & \textbf{Dissemination Level}\footnote{\textbf{PU} = Public, \textbf{CO} = Confidential, only for members of the consortium(including the Commission Services)} & \textbf{Delivery Date} \\
      \hline
        D-1 & Requirement Design Documents & WP-1 & R & CO & 24.12.2016 \\
        \hline
        D-2.1 & Data Model Design & WP-2.1 & R & CO & 13.05.2016 \\
        \hline
        D-2.2 & Domain Model Design & WP-2.2 & R & CO & 13.05.2016 \\
        \hline
        D-3.1 & Renderscript Implementation & WP-3.1 & P & PU & 18.03.2016 \\
        \hline
        D-3.2 & Java Implementation & WP-3.2 & P & PU & 08.04.2016 \\
        \hline
        D-3.3 & OpenCV Implementation & WP-3.3 & P & PU & 12.05.2016 \\
        \hline
    \end{tabularx}
\begin{subsubsection}{Milestones List}

    \begin{tabularx}{\textwidth}{|*{5}{Y|}}
        \hline
        \textbf{Milestone Number} & \textbf{Milestone Name} & \textbf{Work Package(s) Involved} & \textbf{Expected Date} & \textbf{Means of Verification} \\
        \hline
        M1 & Renderscript Application & WP-3.1 & 18.03.2015 & Validated by supervisor \\
        \hline
        M2 & Java Application & WP-3.2 & 08.04.2016 & Validated by supervisor \\
        \hline
        M3 & OpenCV Application & WP-3.3 & 12.05.2016 & Validated by supervisor \\
        \hline
    \end{tabularx}
\end{subsubsection}
\newpage
\begin{subsubsection}{Work Package Descriptions}
  \begin{tabularx}{\textwidth}{|*{6}{Y|}}
    \hline
    \multicolumn{2}{|l|}{\textbf{Work Package Number}}&WP-1&\multicolumn{2}{c|}{\textbf{Start Date:}}&12.10.2015\\
    \hline
    \multicolumn{2}{|l|}{\textbf{Work Package Title}}&\multicolumn{4}{c|}{Research and Requirement Analysis}\\
    \hline
    \multicolumn{2}{|l|}{\textbf{Activity Type}}&\multicolumn{4}{c|}{MGT}\\
    \hline
    \multicolumn{2}{|l|}{\textbf{Participant Number}} &     1   &   2   &   3   &\\
    \hline
    \multicolumn{2}{|l|}{\textbf{Participant Short Name}} &     AA   &   OT   &   UA   &\\
    \hline
    \multicolumn{2}{|l|}{\textbf{Person-weeks per participant:}} &     26 weeks   &   26 weeks   &   26 weeks   &\\
    \hline
  \end{tabularx}

  \fbox{
    \begin{minipage}{37.3em}
      \textbf{Objectives} \\
      \begin{itemize}
      \item Determining the project scope, gathering information about the project,\\
        determining requirements and analysis.\\
      \end{itemize}
    \end{minipage}
  }

  \fbox{
    \begin{minipage}{37.3em}
      \textbf{Description of work} \\
      \textbf{Task-1:} Determine the project scope \\
      \textbf{Task-2:} Research about similar works \\
      \textbf{Task-3:} Making a requirement analyses.\\
      \textbf{Task-5:} Determine the main modules.\\
    \end{minipage}
  }

  \fbox{
    \begin{minipage}{37.3em}
      \textbf{Deliverables} \\
      \begin{itemize}
      \item Documentation of project description, its scope and requirements.\\
      \end{itemize}
    \end{minipage}
  }
\newpage
  \begin{tabularx}{\textwidth}{|*{6}{Y|}}
    \hline
    \multicolumn{2}{|l|}{\textbf{Work Package Number}}&WP-2.1&\multicolumn{2}{c|}{\textbf{Start Date:}}&14.12.2015\\
    \hline
    \multicolumn{2}{|l|}{\textbf{Work Package Title}}&\multicolumn{4}{c|}{Data Model Design}\\
    \hline
    \multicolumn{2}{|l|}{\textbf{Activity Type}}&\multicolumn{4}{c|}{SUPP}\\
    \hline
    \multicolumn{2}{|l|}{\textbf{Participant Number}} &     1   &   2   &   3   &\\
    \hline
    \multicolumn{2}{|l|}{\textbf{Participant Short Name}} &     AA   &   OT   &   UA   &\\
    \hline
    \multicolumn{2}{|l|}{\textbf{Person-weeks per participant:}} &     20 week   &   20 week   &   20 week   &\\
    \hline
  \end{tabularx}

  \fbox{
    \begin{minipage}{37.3em}
      \textbf{Objectives} \\
      \begin{itemize}
      \item Determining data structures that can be commonly used with computer vision algorithms.\\
      \end{itemize}
    \end{minipage}
  }

  \fbox{
    \begin{minipage}{37.3em}
      \textbf{Description of work} \\
      \textbf{Task-1:} Decide data structures \\
      \textbf{Task-2:} Discuss the functionality \\
    \end{minipage}
  }

  \fbox{
    \begin{minipage}{37.3em}
      \textbf{Deliverables} \\
      \begin{itemize}
      \item Data model.\\
      \end{itemize}
    \end{minipage}
  }
\newpage
  \begin{tabularx}{\textwidth}{|*{6}{Y|}}
    \hline
    \multicolumn{2}{|l|}{\textbf{Work Package Number}}&WP-2.2&\multicolumn{2}{c|}{\textbf{Start Date:}}&21.12.2015\\
    \hline
    \multicolumn{2}{|l|}{\textbf{Work Package Title}}&\multicolumn{4}{c|}{Domain Model Design}\\
    \hline
    \multicolumn{2}{|l|}{\textbf{Activity Type}}&\multicolumn{4}{c|}{SUPP}\\
    \hline
    \multicolumn{2}{|l|}{\textbf{Participant Number}} &     1   &   2   &   3   &\\
    \hline
    \multicolumn{2}{|l|}{\textbf{Participant Short Name}} &     AA   &   OT   &   UA   &\\
    \hline
    \multicolumn{2}{|l|}{\textbf{Person-weeks per participant:}} &     19 weeks   &   19 weeks   &   19 weeks   &\\
    \hline
  \end{tabularx}

  \fbox{
    \begin{minipage}{37.3em}
      \textbf{Objectives} \\
      \begin{itemize}
      \item Determine the classes and their attributes.\\
      \end{itemize}
    \end{minipage}
  }

  \fbox{
    \begin{minipage}{37.3em}
      \textbf{Description of work} \\
      \textbf{Task-1:} Determine the classes and their attributes. \\
      \textbf{Task-2:} Determine the relationships among classes \\
    \end{minipage}
  }

  \fbox{
    \begin{minipage}{37.3em}
      \textbf{Deliverables} \\
      \begin{itemize}
      \item Domain model.\\
      \end{itemize}
    \end{minipage}
  }
\newpage
  \begin{tabularx}{\textwidth}{|*{6}{Y|}}
    \hline
    \multicolumn{2}{|l|}{\textbf{Work Package Number}}&WP-3.1&\multicolumn{2}{c|}{\textbf{Start Date:}}&04.01.2016\\
    \hline
    \multicolumn{2}{|l|}{\textbf{Work Package Title}}&\multicolumn{4}{c|}{Renderscript Implementation}\\
    \hline
    \multicolumn{2}{|l|}{\textbf{Activity Type}}&\multicolumn{4}{c|}{SUPP}\\
    \hline
    \multicolumn{2}{|l|}{\textbf{Participant Number}} &     1   &  2   &  3   &\\
    \hline
    \multicolumn{2}{|l|}{\textbf{Participant Short Name}} &     AA   & OT   &  UA   &\\
    \hline
    \multicolumn{2}{|l|}{\textbf{Person-weeks per participant:}} &   10 weeks   & 10 weeks & 10 weeks &\\
    \hline
  \end{tabularx}

  \fbox{
    \begin{minipage}{37.3em}
      \textbf{Objectives} \\
      \begin{itemize}
      \item Implement Harris Corner Detection Algorithm with Renderscript.\\
      \end{itemize}
    \end{minipage}
  }

  \fbox{
    \begin{minipage}{37.3em}
      \textbf{Description of work} \\
      \textbf{Task-1:} Renderscript implementation. \\
    \end{minipage}
  }

  \fbox{
    \begin{minipage}{37.3em}
      \textbf{Deliverables} \\
      \begin{itemize}
      \item Renderscript sample application\\
      \end{itemize}
    \end{minipage}
  }
\newpage
  \begin{tabularx}{\textwidth}{|*{6}{Y|}}
    \hline
    \multicolumn{2}{|l|}{\textbf{Work Package Number}}&WP-3.2&\multicolumn{2}{c|}{\textbf{Start Date:}}&21.03.2016\\
    \hline
    \multicolumn{2}{|l|}{\textbf{Work Package Title}}&\multicolumn{4}{c|}{Java Implementation}\\
    \hline
    \multicolumn{2}{|l|}{\textbf{Activity Type}}&\multicolumn{4}{c|}{SUPP}\\
    \hline
     \multicolumn{2}{|l|}{\textbf{Participant Number}} &     1   &   2   &   3   &\\
    \hline
    \multicolumn{2}{|l|}{\textbf{Participant Short Name}} &     AA   &   OT   &   UA   &\\
    \hline
    \multicolumn{2}{|l|}{\textbf{Person-weeks per participant:}} &     3 weeks   &   3 weeks   &   3 weeks   &\\
    \hline
  \end{tabularx}

  \fbox{
    \begin{minipage}{37.3em}
      \textbf{Objectives} \\
      \begin{itemize}
      \item Implement Harris Corner Detection Algorithm with Java.\\
      \end{itemize}
    \end{minipage}
  }

  \fbox{
    \begin{minipage}{37.3em}
      \textbf{Description of work} \\
      \textbf{Task-1:} Java implementation. \\
    \end{minipage}
  }

  \fbox{
    \begin{minipage}{37.3em}
      \textbf{Deliverables} \\
      \begin{itemize}
      \item Java sample application\\
      \end{itemize}
    \end{minipage}
  }
\newpage
  \begin{tabularx}{\textwidth}{|*{6}{Y|}}
    \hline
    \multicolumn{2}{|l|}{\textbf{Work Package Number}}&WP-3.3&\multicolumn{2}{c|}{\textbf{Start Date:}}&11.04.2016\\
    \hline
    \multicolumn{2}{|l|}{\textbf{Work Package Title}}&\multicolumn{4}{c|}{OpenCV Implementation}\\
    \hline
    \multicolumn{2}{|l|}{\textbf{Activity Type}}&\multicolumn{4}{c|}{SUPP}\\
    \hline
    \multicolumn{2}{|l|}{\textbf{Participant Number}} &     1   &  2   &   3  &\\
    \hline
    \multicolumn{2}{|l|}{\textbf{Participant Short Name}} &     AA   &  OT  &   UA  &\\
    \hline
    \multicolumn{2}{|l|}{\textbf{Person-weeks per participant:}} &   5 weeks   & 5 weeks & 5 weeks &\\
    \hline
  \end{tabularx}

  \fbox{
    \begin{minipage}{37.3em}
      \textbf{Objectives} \\
      \begin{itemize}
      \item Implement Harris Corner Detection Algorithm with OpenCV.\\
      \end{itemize}
    \end{minipage}
  }

  \fbox{
    \begin{minipage}{37.3em}
      \textbf{Description of work} \\
      \textbf{Task-1:} OpenCV implementation. \\
    \end{minipage}
  }

  \fbox{
    \begin{minipage}{37.3em}
      \textbf{Deliverables} \\
      \begin{itemize}
      \item OpenCV sample application\\
      \end{itemize}
    \end{minipage}
  }
\newpage

  \begin{tabularx}{\textwidth}{|*{6}{Y|}}
    \hline
    \multicolumn{2}{|l|}{\textbf{Work Package Number}}&WP-4&\multicolumn{2}{c|}{\textbf{Start Date:}}&18.01.2016\\
    \hline
    \multicolumn{2}{|l|}{\textbf{Work Package Title}}&\multicolumn{4}{c|}{Testing}\\
    \hline
    \multicolumn{2}{|l|}{\textbf{Activity Type}}&\multicolumn{4}{c|}{SUPP}\\
    \hline
     \multicolumn{2}{|l|}{\textbf{Participant Number}} &     1   &   2   &   3   &\\
    \hline
    \multicolumn{2}{|l|}{\textbf{Participant Short Name}} &     AA   &   OT   &   UA   &\\
    \hline
    \multicolumn{2}{|l|}{\textbf{Person-weeks per participant:}} &     18 weeks   &   18 weeks   &   18 weeks   &\\
    \hline
  \end{tabularx}

  \fbox{
    \begin{minipage}{37.3em}
      \textbf{Objectives} \\
      \begin{itemize}
      \item Test applications on different devices with different resolutions.\\
      \end{itemize}
    \end{minipage}
  }

  \fbox{
    \begin{minipage}{37.3em}
      \textbf{Description of work} \\
      \textbf{Task-1:} Compare the performances of each implementation. \\
    \end{minipage}
  }

  \fbox{
    \begin{minipage}{37.3em}
      \textbf{Deliverables} \\
      \begin{itemize}
      \item Comparison results\\
      \end{itemize}
    \end{minipage}
  }
\end{subsubsection}
\newpage
\begin{subsubsection}{Summary Effort Table}
  \begin{tabularx}{\textwidth}{|*{9}{Y|}}
   \hline
   \textbf{P. short name} & \textbf{WP-1} & \textbf{WP-2.1} & \textbf{WP-2.2} & \textbf{WP-3.1} & \textbf{WP-3.2} & \textbf{WP-3.3} & \textbf{WP-4} & \textbf{Total person weeks} \\
   \hline
   AA & 26 & 20 & 19 & 10 & 3 & 5 & 18 & 101 \\
   \hline
   OT & 26 & 20 & 19 & 10 & 3 & 5 & 18 & 101 \\
   \hline
   UA & 26 & 20 & 19 & 10 & 3 & 5 & 18 & 101 \\
  \hline
 \end{tabularx}
\end{subsubsection}
\newpage
\begin{landscape}
\begin{subsection}{Pert Diagram}
  \newline
  \vspace{5cm}
  \newline
  \begin{center}
    \tikzset{
      every shadow/.style={
        fill=none,
        shadow xshift=0pt,
        shadow yshift=0pt}
    }
   \smartdiagramset{
     border color=black,
     back arrow disabled=true,
     arrow line width=1pt,
     uniform color list=white for 5 items,
     uniform arrow color=true,
     arrow color=black,
     font=\medium,
     text width=2.5cm,
     module minimum width=2.5cm,
     module minimum height=1.5cm,
     arrow tip=stealth,
     module x sep=3.75,
     back arrow distance=0.75,
     additions={
     additional item offset=0.85cm,
     additional item border color=black,
     additional item fill color=white,
     additional item width=2.5cm,
     additional connections disabled=true,
     additional arrow color=black,
     additional arrow tip=stealth,
     additional arrow line width=1pt,
     additional arrow style=[-stealth’
     }
   }
  \smartdiagramadd[flow diagram:horizontal]{%
  Search of Similar Research,Domain and Data Model,Renderscript Implementation,Sample Apps,Result%
  }{%
  below of module1/Research and Requirement Analysis, below of module3/Java Implementation,
  above of module3/OpenCV Implementation, below of module4/Testing%
  }
  \smartdiagramconnect{{]-latex’}}{additional-module1/module2}
  \smartdiagramconnect{{]-latex’}}{additional-module2/module4}
  \smartdiagramconnect{{]-latex’}}{additional-module3/module4}
  \smartdiagramconnect{{]-latex’}}{additional-module4/module4}
  \end{center}
\end{subsection}
\end{landscape}
\newpage
\begin{subsection}{Risk Management}
  \paragraph{}{%
  The development team was unexperienced on computer vision field, so some learning
  period was required in multiple phases of the project. Especially digesting Harris Corner Detection Algorithm
  and implementing Renderscript and OpenCV versions of it were formidable.
  For this reason it was crucial to comply with the schedule.\\
  }
\end{subsection}
\end{section}

\newpage

\begin{section}{Analysis and Design}
  \begin{subsection}{The Process Model and Its Particular Adaptation}
    \paragraph{}{According to R. S. Pressman \cite{pressman} there are several process model types including linear sequential model, prototype model, evolutionary models and etc.
    In real life most them are used according to the project necessities and requirements.}
    \paragraph{}{reVision is being developed using incremental model which is an evolutionary model.
    Every release is published after a work cycle which includes all the phases of the software development process.
    Each release extends the project with at least one implementation with its sample application.}
    \paragraph{}{Incremental model is suitable for this project because the project team is not experienced in
    computer vision field. The incremental model gives the option of refactoring the design and evolving the project
    through time to the team, which other models would not give.}
    \paragraph{}{Development cycle can simply be clarified with following steps; Requirement analysis, Planning, Design,
    Implementation and Testing. Every development iteration of reVision include each one of those steps.}
  \end{subsection}
  \newpage
  \begin{subsection}{Functional and Non-Functional Requirements}
    \begin{subsubsection}{Functional Requirements}
      \paragraph{Corner Detection:}{
        Each implementation detects corners on the video input.
      }
      \paragraph{Performance Measurement:}{
        Sample applications calculate the performance in terms of frame per second(FPS).
      }

      \newline
      \vspace{1cm}
      \newline
      \lstset{language=C++,
                      caption={Corner Response Kernel},
                      captionpos=b,
                      backgroundcolor=\color{black!5},
                      keywordstyle=\color{new-blue},
                      stringstyle=\color{red},
                      commentstyle=\color{green},
                      basicstyle=\footnotesize,% basic font setting
                      morecomment=[l][\color{magenta}]{\#}
      }
      {\ttfamily
      \begin{lstlisting}
        float __attribute__((kernel)) cornerResponse(const float in,
                                                     uint32_t x,
                                                     uint32_t y) {
            float Ixx = rsGetElementAt_float(allIxx, x, y);
            float Iyy = rsGetElementAt_float(allIyy, x, y);
            float Ixy = rsGetElementAt_float(allIxy, x, y);

            float cornerResponse =
                (Ixx * Iyy - Ixy * Ixy -
                c * (Ixx + Iyy) * (Ixx + Iyy));
            if(cornerResponse < -harrisThreshold ||
               cornerResponse > harrisThreshold) {
                rsSetElementAt_float(covImg, cornerResponse, x, y);
            } else {
                rsSetElementAt_float(covImg, 0, x, y);
            }
            return in;
        }
      \end{lstlisting}
      }

      \lstset{language=C++,
                      caption={Non-maximal Suppression Kernel},
                      captionpos=b,
                      backgroundcolor=\color{black!5},
                      keywordstyle=\color{new-blue},
                      stringstyle=\color{red},
                      commentstyle=\color{green},
                      basicstyle=\footnotesize,% basic font setting
                      morecomment=[l][\color{magenta}]{\#}
      }
      {\ttfamily
      \begin{lstlisting}
        float __attribute__((kernel)) nonMaxSuppression(const float in,
                                                        uint32_t x,
                                                        uint32_t y) {
            float tmp = rsGetElementAt_float(covImg, x, y);
            for(ky = -1*nonMaxRadius;
                tmp != 0 && ky <= nonMaxRadius;
                ky++) {
                    for(kx = -1*nonMaxRadius;
                        kx <= nonMaxRadius;
                        kx++) {
                            if(rsGetElementAt_float(covImg,
                                                    x + kx,
                                                    y + ky) > tmp) {
                                tmp = 0;
                                break;
                            }
                    }
            }
            rsSetElementAt_float(covImg, tmp, x, y);
            return in;
        }
      \end{lstlisting}
      }
    \end{subsubsection}
    \newpage
    \begin{subsubsection}{Non-Functional Requirements}
      \paragraph{Performance Requirements:}{
        The response time of commonly used solutions were the thing that triggered the creation of reVision project.
        Without reasonable performance, reVision could not be considered successful.
        The performance is the single most important non-functional requirement of reVision.
      }
      ​
      \paragraph{Reliability Requirements:}{
        The responses of reVision implementations must be correct and precise.
      }
    \end{subsubsection}
  \end{subsection}

  \begin{subsection}{Use Cases}
    reVision is an experimental project to provide analysis results of Computer Vision implementations of different technologies.
    It is not intended to be used by end users. Therefore there is no use case to share.

  \end{subsection}
\end{section}
\newpage
\begin{section}{Solution/Product \& Results}

  \paragraph{}{
    reVision is planned to compare Renderscript with OpenCV and Java which are commonly used technologies for Computer
    Vision. Renderscript is a framework developed by Google to make computationally intensive tasks at high performance
    on Android. However, it is not used by the most of Android developers. reVision strives to create a knowledge base
    for developers to help which technology to use. It is an analysis report that makes comparison on different
    Android devices with different resolutions.
  }

  \begin{figure}[h]
    \centering
    \includegraphics[scale=0.60]{graph.png}

    \caption{Performance of different devices}
    \label{fig:mesh1}
  \end{figure}

  \paragraph{}{
    Harris Corner Detection Algorithm is implemented with Renderscript, OpenCV and Java. OpenCV is chosen for being
    the most common library in this field. It is reasonable to compare Renderscript with it. And Java is the main
    language of Android framework.
  }
  \newpage

\paragraph{}{

  \begin{figure}[h]
    \centering
    \includegraphics[scale=0.25]{renderscript.png}
    \caption{Renderscript Demo}
    \label{fig:mesh2}
  \end{figure}

  \begin{figure}[h]
    \centering
    \includegraphics[scale=0.25]{opencv.png}
    \caption{OpenCV Demo}
    \label{fig:mesh3}
  \end{figure}

  \begin{figure}[h]
    \centering
    \includegraphics[scale=0.25]{java.png}
    \caption{Java Demo}
    \label{fig:mesh4}
  \end{figure}
}
\end{section}
\vfill
\newpage
\begin{section}{Related Work/Similar Solutions}
\paragraph{}{
There are several libraries about the vision. One of the most significant of them is OpenCV.

OpenCV is released under a BSD license and hence it’s free for
both academic and commercial use. It has C++, C, Python and Java interfaces and supports Windows, Linux, Mac OS,
iOS and Android. OpenCV was designed for computational efficiency and with a strong focus on real-time applications.
Written in optimized C/C++, the library can take advantage of multi-core processing. Enabled with OpenCL, it can
take advantage of the hardware acceleration of the underlying heterogeneous compute platform. Adopted all around
the world, OpenCV has more than 47 thousand people of user community and estimated number of downloads exceeding 9
million. Usage ranges from interactive art, to mines inspection, stitching maps on the web or through advanced
robotics \cite{opencv}.
}

  \paragraph{}{
Another computer vision library is the Qualcomm's FastCV\textsuperscript{TM} \cite{fastcv}. FastCV\textsuperscript{TM} library offers a
mobile-optimized computer vision (CV) library which includes the most frequently used vision processing functions
for use across a wide array of mobile devices, even mass-market handsets. Middleware developers can use FastCV to
build the frameworks needed by developers of computer vision apps; Qualcomm's Augmented Reality (AR) SDK is a good
example. Developers of advanced CV application can also use FastCV functions directly in their application. FastCV will
enable you to add new user experiences into your camera-based apps like:
\begin{itemize}
  \item gesture recognition
  \item face detection, tracking and recognition
  \item text recognition and tracking
  \item augmented reality
\end{itemize}

}

\paragraph{}{
Also Cuda\textsuperscript{TM} \cite{cuda} can be mentioned as a similar work. CUDA\textsuperscript{TM} is a parallel computing platform and programming model invented by
NVIDIA. It enables dramatic increases in computing performance by harnessing the power of the graphics processing unit
(GPU). To use CUDA on your system, you will need the following:
\begin{itemize}
  \item Android development device with a CUDA-capable GPU
  \item A supported version of Linux to cross-compile
  \item NVIDIA CodeWorks for Android with CUDA support
\end{itemize}
}

\end{section}

\newpage

\begin{section}{Impact}

\paragraph{}{
In the impact section, realistic constraints of the project will be mentioned. Besides being a thesis project,
reVision is a nonprofit project as a matter of course, aims to make the Android world a better place for computer vision
application developers. Thus, there are several constraints about the project. When compared to similar works,
reVision is maintained by students instead of groups of experienced developers or big companies.
So, this situation limits the scope of the project, but it also motivates its participants about the dream of
taking part in a project that can be a milestone as well.

}

\paragraph{}{
  reVision is also an opportunity to learn new technologies and methodologies for its developers. It focuses on the
  computer vision and its applications using mobile devices, and data-parallel GPU computation. So, project team participants
  get a chance to improve themselves in these topics. The project will help the developers to understand the formal procedures
  of writing formal reports. Nearly all the software engineering processes are revisited.
}

\paragraph{}{
  All in all,  reVision is a hardware independent computer vision library with companion sample applications
  for Android which will be an open source and GPU-ready, and strives to make Android developers more productive
  on computer vision projects.

}

\end{section}

\newpage
          \begin{thebibliography}{9}
          \bibitem{renderscript_release}
            Introducing Renderscript -
              http://android-developers.blogspot.com.tr/2011/02/introducing-renderscript.html

          \bibitem{renderscript_api}
            Renderscript - http://developer.android.com/intl/es/guide/topics/renderscript/compute.html

          \bibitem{harris_article}
            Harris, C. and Stephens, M. 1988. Acombined corner and edge detector. In Fourth Alvey
            Vision Conference, Manchester, UK, pp. 147–151.

            \bibitem{opencv}
              OpenCV -  http://opencv.org/

          \bibitem{pressman}
            Roger S. Pressman, Software Engineering A Practitioner's Approach,
            McGraw-Hill, 2001, pp. 26-39

          \bibitem{fastcv}
            FastCV - https://developer.qualcomm.com/software/fastcv-sdk

          \bibitem{cuda}
          Cuda - http://docs.nvidia.com/gameworks/content/technologies/mobile/cuda\_android\_main.htm

          \end{thebibliography}

\end{document}
